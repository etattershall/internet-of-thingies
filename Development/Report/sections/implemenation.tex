% TODO:
% Line about PI talks to PI
% Use 'IoT'
% Mention outreach? - perhaps to introduce the graph section
% Sort out topics - in code!
% - What is broker-services/hello?
% - Don't post the time in the last will
% - Order of time and status
% Ensure that what is written here about Bluetooth working with our setup if a set of devices is provided is correct
% Sort out tenses
% Confirm 'suffix' is a correct term for final topic in MQTT?
% Pictures of the equipment

We implemented a test bed for future development of an Internet of Things security protocol. Our proof-of-concept used standard IoT components to produce a system that moved data over the different levels of our architecture, demonstrating common features of a large scale infrastructure.

% ✔ We wanted to make a prototype for use as a test bed (or proof-of-concept) for future development of an IoT security protocol, blah.
% Also used for outreach activities, blah
% ✔ Chose to use off-the-shelf component for maximum applicability, blah


\subsection{Equipment}
\paragraph{}
For Edge Devices, we used Arduino Unos which are inexpensive (£25-30), low powered and portable making them a popular choice for use with IoT. With 14 inputs and outputs (analog and digital), these microcontrollers can be loaded with pre-compiled programs to interact with sensors. To communicate with Smart Agents, we used Serial over either USB or Bluetooth (through the addition of a HC-05 Bluetooth module - see Figure \ref{fig:HC-05}).


\begin{figure}
    \centering
    \includegraphics[width=0.5\textwidth]{HC05.jpg}
    \caption{The HC-05 chip for serial over Bluetooth. It costs roughly £4 and has a range of about 10m. Requiring a pin to pair the devices, it can have the role of either slave (wait for connection) or master (search for device to connect to).}
    \label{fig:HC-05}
\end{figure}

\paragraph{}
As Smart Agents, we used the Raspberry PI 3 which is a single board computer. Communication with Edge Devices and the Cloud was possible over USB, Ethernet, WiFi and Bluetooth without any additional adapters. The PI can run PC scale applications headless on its Linux operating system. Its low price, small form factor and low power requirement make it popular for IoT.

\paragraph{}
We used STFC's SCD Cloud for access to a virtual machine running Scientific Linux 7. A common LAN enabled TCP/IP communication between this machine and the Smart Agents and the web application.

% ✔ We used Arduino Unos (£25-30) as edge devices. They have X inputs and outputs (analog and digital) and can be loaded with pre-compiled programs.
% ✔ HC-05 chip for Bluetooth option (picture of HC-05, range=10m?, cost=£4...)
% ✔ Raspberry pi as smart agents. Stats – 1GB RAM, can be run headless, use linux...
% ✔ SCD Cloud machine (Scientific Linux 7) hugely useful, blah.

\subsection{Setup}

\begin{figure}
    \centering
    \includegraphics[width=0.5\textwidth]{Architecture.png}
    \caption{The architecture}
    \label{fig:architecture}
\end{figure}

\paragraph{}
The Cloud VM was setup as the MQTT broker using Mosquitto. There were three endpoints contacting it: Smart Agents, Broker Services and the web application. Using the Paho MQTT client library for Python, the Smart Agents acted as a gateway to the Cloud for any connected Edge Devices. Broker Services ran on the Cloud as a Python application (also using Paho) to publish a list of currently connected devices for discovery. A Flask module (Flask-MQTT) served a web application, allowing a user to subscribe to topics for an overview of the data flows in the network.

\paragraph{}
There were several problems with using Bluetooth. In our initial setup, the HC-05s were configured in slave mode with names according to a pattern. The Smart Agent would periodically scan for new devices according to the pattern and then attempt to connect. However, with only one Bluetooth chip, each PI could not send/receive while scanning so communications experienced a substantial delay (roughly 20s). To counter this, we used a fixed list of Edge Devices per Smart Agent which were always available. While seemingly restrictive, this would work adequately for a variety of use cases - such as a central Smart Agent controlling statically placed sensors in a house or vehicle. Additionally, the number of devices in each piconet was restricted to seven. This didn't affect our prototype but may be an issue for scaling in a real setup.

\paragraph{}
Depending on the use case, different protocols and technologies may be more applicable. If Edge Devices are relatively fixed, our serial protocol worked over USB - where there was no delay to scanning but was it was not wireless. Technologies that are - such as Zigbee - can connect a larger number of devices but require bespoke hardware at both ends. Our technique of sending a JSON encoded packet could work on any text communication protocol - we implemented it over USB and Bluetooth serial.

% ✔ Each edge device is identified by a unique ID which is a random number
% ✔ Communication between Smart Agent and Edge via Bluetooth.
% What is Bluetooth /why is it a good option for communicating with low memory/power/connectivity devices? PAIRING!
% ✔ Smart agents run Piduino to interface between MQTT Pub/Sub and the inputs and outputs in the edge
% ✔ Broker (mosquitto) runs on cloud, as does telemetry and monitoring applications (broker services) and a flask app.
% Pretty diagrams and pictures of the equipment
% ✔ Diagram with the pretty logos on it I made for the pre-coffee talk. I’ll go find it. The programming


\subsection{Data}

% Pretty graph of some data collected
% Graph, and also some brief waffle about what’s on the graph.
% Picture of the Arduino/raspberry pi setup


\begin{center}
    \begin{tikzpicture}
    \begin{axis}[
        title={Light Intensity over the 28th and 29th June},
        xlabel={Time [HH:MM]},
        ylabel={Arduino Reading [1024/5V]},
        xlabel near ticks,
        date coordinates in=x,
        xticklabel style={rotate=90,
                          anchor=near xticklabel},
        xticklabel=\hour:\minute,
        legend pos=outer north east,
        legend entries={Emma,Callum},
    ]
        \addplot [blue, mark=none] table [x=Datetime,y=Emma, col sep=tab] {LDR_dates.tsv};
        \addplot [orange, mark=none] table [x=Datetime,y=Callum, col sep=tab] {LDR_dates.tsv};
    \end{axis}
    \end{tikzpicture}
\end{center}
