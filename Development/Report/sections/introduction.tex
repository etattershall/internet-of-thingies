As computers and microcontrollers become smaller and more affordable, they have begun to find their way into everyday objects, forming a distributed network of smart devices called and Internet of Things. The owners of these smart devices are able to control and receive data from their devices remotely. There are potential applications for this technology in the areas of home automation \cite{openhab}, citizen science \cite{radiation}\cite{flooddetection} and smart cities \cite{smartcities}.

\paragraph{}
However, recent rapid development has led to security concerns as manufacturers and researchers race to be the first to implement their products. Securing these systems is made more difficult by the devices’ limitations in terms of memory, processor speed and connectivity; it is sometimes simply not possible to store a long cryptographic key, or apply a time-expensive algorithm for encryption, or transmit a long key over the network. Lapses in security implementation have allowed the spread of IoT specific malware such as the Mirai botnet, which takes advantage of factory default username-password pairs used to secure home routers and IP cameras \cite{mirai}.

\paragraph{}
As we build larger, more complex and more distributed systems, the question of architecture and security becomes ever more important. The EU project mF2C, has been set up to explore the issues of architecture, security and implementation in large, distributed IoT systems by considering a number of use cases. These use cases include emergency situation management in smart cities, enriched navigation for vessels at sea and smart hub services for public environments, such as airports, hospitals and shopping centres \cite{mf2cwebsite}. This last case, in the context of a Smart Airport in which consumers are able to check up-to-date flight information, find their friends and receive special offers is of particular interest to us and we have used it to inform our thinking in the areas of architecture and security during this project.
 
\paragraph{}
In this report, we present a hierarchical architecture in which more powerful devices can act as hubs to ensure the security and privacy needs of smaller devices.  We have developed a prototype architecture using off-the-shelf components for use as a test bed for building a comprehensive internet of things security protocol.
