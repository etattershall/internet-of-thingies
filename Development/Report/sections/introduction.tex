internet of things: as computers and microcontrollers become smaller and more affordable, they will begin to find their way into everyday objects, forming a distributed network of smart devices. Users will be able to control and receive data from their devices remotely. There are applications in the areas of home automation, citizen science and smart cities, to name just a few examples.

However, the recent rapid development of this technology has led to security concerns, as manufacturers and researchers race to be the first to implement their products. Securing these systems is made more difficult by the devices’ limitations in terms of size, processor speed and connectivity; it is sometimes simply not possible to store a long cryptographic key, or apply a time-expensive algorithm for encryption, or transmit a long key over the network. Lapses in security implementation have allowed the spread of IoT specific malware such as the Mirai botnet, which takes advantage of factory default username-password pairs used to secure home routers and IP cameras.

As we build larger, more complex and more distributed systems, the question of architecture and security becomes ever more important. The EU project mF2C, has been set up to explore these issues by looking at a number of use cases. These include...

...
 including a Smart Airport in which users are able to check flight information, find their friends and receive advertising from their phones, ...
 
In this report, we present a hierarchal architecture in which more powerful devices can act as hubs to ensure the security and privacy needs of smaller devices.  We have built and coded a prototype architecture using off-the-shelf components for use as a test bed for developing a comprehensive internet of things security protocol.